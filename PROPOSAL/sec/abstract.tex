

\newpage
\section{Abstract}
\addcontentsline{toc}{section}{ABSTRACT}

This report presents the development process, design decisions, and outcomes of our Tetris game project. Tetris, a classic tile-matching puzzle video game, has been a popular choice for both casual gamers and enthusiasts since its inception. This project aimed to create a modern and engaging implementation of Tetris while exploring key aspects of game development, Object-Oriented Principles, graphics rendering, user interaction, and algorithm design.

The project involved designing and implementing the game mechanics, user interface, scoring system, and visual elements of Tetris. Leveraging the capabilities of SFML(Simple and Fast Multimedia Library), we built a playable version of Tetris that captures the essence of the original game while adding visual enhancements and a user-friendly interface. 

In this report, we delve into the technical details of our game's architecture and discuss the challenges we encountered during development. We describe our approach to handling user input, managing game state, rendering graphics, and optimizing performance. Additionally, we highlight the design considerations that went into creating an intuitive user experience, including responsive controls and visual feedback.

Throughout the project, we had the opportunity to apply principles learned in our Object Oriented Programming course, gaining hands-on experience in software design, algorithms, and teamwork. We collaborated closely, leveraging each team member's strengths to overcome obstacles and make informed design choices. Our project supervisors, Mr. Daya Sagar Baral and Mr. Rajad Shakya, provided valuable guidance and feedback, ensuring the project's alignment with our learning objectives.

As a result of our efforts, we have successfully developed a functional and enjoyable Tetris game that showcases our technical skills and creativity. This project not only enhanced our understanding of game development but also provided us with a platform to apply theoretical knowledge to a practical context. We hope that our Tetris game brings joy to players.
At last, we learned to use SFML library in our C++ projects and learned a lot about GIT and Version Control Systems while working in a team\\
