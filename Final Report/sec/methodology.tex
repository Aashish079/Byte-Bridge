	\newpage
\section{METHODOLOGY}
\hspace{5mm}This  project aimed to develop a Tetris game using Object-Oriented Programming principles and the SFML library for graphics. The project was implemented using the CLion Integrated Development Environment and utilized CMake for cross-platform build support. Version control and collaborative coding was facilitated through Git and GitHub.\\
For detailed history of development of this game,\\
Link to our GitHub Repository:\\
\textit{https://github.com/Aashish079/Tetris}\\



To ensure a structured and maintainable codebase, we have designed the class hierarchy for the game. Different classes are created by inheriting from the parent class State, to represent game objects such as the GamePlayState class, GameOverState class, HighScoreState tetrominoes, and FileManager. We have employed OOP concepts such as encapsulation, inheritance, and polymorphism to achieve modularity and code reusability in this project. The game logic is implemented within the GamePlayState class. Each class have private and public member variables and member functions that handle specific aspects of the game, such as moving and rotating tetrominoes, checking for line clears, and updating the score. The SFML library is utilized for rendering graphics and handling user input.\\

To ensure efficient functioning, smooth transistioning and good development experience we have used starter code of SFML named TheStateMachine which stores and renders different states such as MainMenuState, GameplayState and GameOverState.\\

To facilitate collaboration among team members, we utilized Git and GitHub for version control. We had total of 7 different branches for different features. Regular commits were performed to the repository and timely pull request were initiated from different branches into main branch. GitHub's issue tracking feature was also utilized to manage and prioritize tasks. Proper documentation was maintained to provide an overview of the project structure, class hierarchy, and function specifications..\\

We tried our best to take the most systematic and effiecient development \\approach throughout the development of this project.\\
