\newpage
\section{LITERATURE SURVEY}

\hspace{5mm}It all began with a puzzle-loving software engineer named Alexey Pajitnov, who created "Tetris" in 1984 while working for the Dorodnitsyn Computing Centre of the Soviet Academy of Sciences, a research and development center in Moscow created by the government.
Pajitnov was inspired by a puzzle game called "pentominoes," in which different wooden shapes made of five equal squares are assembled in a box. Pajitnov imagined the shapes falling from above into a glass, with players controlling the shapes and guiding them into place. Pajitnov adapted the shapes to four squares each and programmed the game in his spare time, dubbing it "Tetris." The name combined the Latin word "tetra" — the numerical prefix "four," for the four squares of each puzzle piece — and "tennis," Pajitnov's favorite game.
And when he shared the game with his co-workers, they started playing it — and kept playing it and playing it. These early players copied and shared "Tetris" on floppy disks, and the game quickly spread across Moscow. When Pajitnov sent a copy to a colleague in Hungary, it ended up on display in a software exhibit at the Hungarian Institute of Technology, where it came to the attention of Robert Stein, owner of Andromeda Software Ltd., who was visiting the exhibit from the United Kingdom.
"Tetris" intrigued Stein. He tracked down Pajitnov in Moscow, but ultimately the game's fate lay in the hands of a new Soviet agency, Elektronorgtechnica (Elorg), created to oversee foreign distribution of Soviet-made software. Elorg licensed the game to Stein, who then licensed it to distributors in the U.S. and the U.K. — Spectrum HoloByte and Mirrorsoft Ltd — The New York Times reported in 1988. According to the Times, "Tetris" was the first software created in the Soviet Union to be sold in America.
In Brown's book, the unusual story of "Tetris" is interwoven with an exploration of gaming: why people do it, how it changes them and how it brings people together. Pajitnov himself began this journey simply because he loved games and puzzles and wanted to share them with the world. And in the process, "Tetris" took on a life of its own.